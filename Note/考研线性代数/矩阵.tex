\ifx\allfiles\undefined
\documentclass[8pt a4paper,oneside,UTF8]{ctexbook} 
\usepackage{amsmath}   % 数学公式
\usepackage[dvipsnames]{xcolor}
\usepackage{amsthm}    % 定理环境
\usepackage{amssymb}   % 更多公式符号
\usepackage{graphicx}  % 插图
\usepackage{mathrsfs}  % 数学字体
\usepackage{enumitem}  % 列表
\usepackage{geometry}  % 页面调整
\usepackage{unicode-math}
\usepackage{extarrows}
\usepackage{subfigure}
\usepackage{extarrows}
\usepackage{footnote}
\usepackage{svg}
\usepackage{lmodern}
\usepackage{anyfontsize}
\usepackage[colorlinks,linkcolor=black]{hyperref}
\usepackage{supertabular}
\usepackage{tcolorbox}
\usepackage{ulem}
\usepackage{framed}
\usepackage{float}
\usepackage{microtype}
\newcommand{\arccot}{\mathrm{arccot}\,}
\tcbuselibrary{breakable}
\tcbuselibrary{most}
\newcounter{problemname}

\newenvironment{solution}{\par\noindent\textbf{解答. }}{\par}
\newenvironment{note}{\par\noindent\textbf{题目\arabic{problemname}的注记. }}{\par}
\definecolor{shadecolor}{RGB}{241, 241, 255}
\newenvironment{problem}{\begin{shaded}\stepcounter{problemname}\par\noindent\textbf{题目\arabic{problemname}. }}{\end{shaded}\par}
\allowdisplaybreaks
\graphicspath{ {figure/},{../figure/}, {config/}, {../config/} }  % 配置图形文件检索目录
\linespread{1.5} % 行高

% 页码设置
\geometry{top=25.4mm,bottom=25.4mm,left=20mm,right=20mm,headheight=2.17cm,headsep=4mm,footskip=12mm}

% 设置列表环境的上下间距
\setenumerate[1]{itemsep=5pt,partopsep=0pt,parsep=\parskip,topsep=5pt}
\setitemize[1]{itemsep=5pt,partopsep=0pt,parsep=\parskip,topsep=5pt}
\setdescription{itemsep=5pt,partopsep=0pt,parsep=\parskip,topsep=5pt}

% 定理环境
% ########## 定理环境 start ####################################

% 定义单独编号,其他四个共用一个编号计数 这里只列举了五种,其他可类似定义(未定义的使用原来的也可)
\newtcbtheorem[auto counter, number within=section, list type=subsubsection, list inside=toc]{defn}{定义}
{
    colback=green!5,colframe=green!35!black,fonttitle=\bfseries, title={Comment \thetcbcounter}, list entry={Comment \thetcbcounter\quad}, %标题
    breakable, %支持跨页
    before upper={\parindent10pt\noindent},  % 支持缩进。\noindent:首行不缩进
    % left = 2mm, %文字离线框左边的边距
    % right = 1mm,%同上
    % top = 1mm,%同上
    % bottom = 1mm,%同上
    % arc is angular = 1mm, % 棱角线框
    % sharp corners, % 直角线框
    % enhanced,frame hidden, % 隐藏线框
    % enhanced, drop fuzzy shadow,  % 显示阴影
}
{def}

\newtcbtheorem[auto counter, number within=section, list type=subsubsection, list inside=toc]{lemma}{引理}
{
    colback=SeaGreen!10!CornflowerBlue!10,colframe=RoyalPurple!55!Aquamarine!100!,fonttitle=\bfseries, title={Comment \thetcbcounter}, list entry={Comment \thetcbcounter\quad}, %标题
    breakable, %支持跨页
    before upper={\parindent10pt\noindent},  % 支持缩进。\noindent:首行不缩进
    % left = 2mm, %文字离线框左边的边距
    % right = 1mm,%同上
    % top = 1mm,%同上
    % bottom = 1mm,%同上
    % arc is angular = 1mm, % 棱角线框
    % sharp corners, % 直角线框
    % enhanced,frame hidden, % 隐藏线框
    % enhanced, drop fuzzy shadow,  % 显示阴影
}
{lem}
\newtcbtheorem[auto counter, number within=section, list type=subsubsection, list inside=toc]{them}{定理}
{
    colback=Salmon!20, colframe=Salmon!90!Black,fonttitle=\bfseries, title={Comment \thetcbcounter}, list entry={Comment \thetcbcounter\quad}, %标题
    breakable, %支持跨页
    before upper={\parindent10pt\noindent},  % 支持缩进。\noindent:首行不缩进
    % left = 2mm, %文字离线框左边的边距
    % right = 1mm,%同上
    % top = 1mm,%同上
    % bottom = 1mm,%同上
    % arc is angular = 1mm, % 棱角线框
    % sharp corners, % 直角线框
    % enhanced,frame hidden, % 隐藏线框
    % enhanced, drop fuzzy shadow,  % 显示阴影
}
{them}
\newtcbtheorem[auto counter, number within=section, list type=subsubsection, list inside=toc]{criterion}{注}
{
    colback=CornflowerBlue!10,colframe=RoyalPurple!55!Aquamarine!100!,fonttitle=\bfseries, title={Comment \thetcbcounter}, list entry={Comment \thetcbcounter\quad}, %标题
    breakable, %支持跨页
    before upper={\parindent10pt\noindent},  % 支持缩进。\noindent:首行不缩进
    % left = 2mm, %文字离线框左边的边距
    % right = 1mm,%同上
    % top = 1mm,%同上
    % bottom = 1mm,%同上
    % arc is angular = 1mm, % 棱角线框
    % sharp corners, % 直角线框
    % enhanced,frame hidden, % 隐藏线框
    % enhanced, drop fuzzy shadow,  % 显示阴影
}
{cri}

\newtcbtheorem[auto counter, number within=section, list type=subsubsection, list inside=toc]{corollary}{推论}
{
    colback=Emerald!10,colframe=cyan!40!black,fonttitle=\bfseries, title={Comment \thetcbcounter}, list entry={Comment \thetcbcounter\quad}, %标题
    breakable, %支持跨页
    before upper={\parindent10pt\noindent},  % 支持缩进。\noindent:首行不缩进
    % left = 2mm, %文字离线框左边的边距
    % right = 1mm,%同上
    % top = 1mm,%同上
    % bottom = 1mm,%同上
    % arc is angular = 1mm, % 棱角线框
    % sharp corners, % 直角线框
    % enhanced,frame hidden, % 隐藏线框
    % enhanced, drop fuzzy shadow,  % 显示阴影
}
{cor}
% colback=red!5,colframe=red!75!black

% ######### 定理环境 end  #####################################

% ↓↓↓↓↓↓↓↓↓↓↓↓↓↓↓↓↓ 以下是自定义的命令  ↓↓↓↓↓↓↓↓↓↓↓↓↓↓↓↓

% 用于调整表格的高度  使用 \hline\xrowht{25pt}
\newcommand{\xrowht}[2][0]{\addstackgap[.5\dimexpr#2\relax]{\vphantom{#1}}}

% 表格环境内长内容换行  
\newcommand{\tabincell}[2]{\begin{tabular}{@{}#1@{}}#2\end{tabular}}

% 使用\linespread{1.5} 之后 cases 环境的行高也会改变,重新定义一个 ca 环境可以自动控制 cases 环境行高
\newenvironment{ca}[1][1]{\linespread{#1} \selectfont \begin{cases}}{\end{cases}}
% 和上面一样
\newenvironment{vx}[1][1]{\linespread{#1} \selectfont \begin{vmatrix}}{\end{vmatrix}}

\def\d{\textup{d}} % 直立体 d 用于微分符号 dx
\def\R{\mathbb{R}} % 实数域
\newcommand{\bs}[1]{\boldsymbol{#1}}    % 加粗,常用于向量
\newcommand{\ora}[1]{\overrightarrow{#1}} % 向量

% 数学 平行 符号
\newcommand{\pll}{\kern 0.5em/\kern -0.8em /\kern 0.5em}

% 用于空行\myspace{1} 表示空一行 填 2 表示空两行  
\newcommand{\myspace}[1]{\par\vspace{#1\baselineskip}}

\begin{document}
\begin{sloppypar}
    % \input{../config/cover} 
    \else
    \fi
    %  ############################ 正文部分
    \chapter{矩阵}
    \section{矩阵的概念}
    \subsection{矩阵的定义}
    \begin{defn}{矩阵的定义}{}
        由$m \times n$个数 $a_{ij}(i=1,2,\cdots,m;j=1,2,\cdots,n)$排成的\textbf{$m$行$n$列}的矩形表格
        $$
            \begin{bmatrix}
                a_{11} & a_{12} & \cdots & a_{1n} \\a_{21}&a_{22}&\cdots&a_{2n}\\\vdots&\vdots&&\vdots\\a_{m1}&a_{m2}&\cdots&a_{mn}
            \end{bmatrix}
        $$
        称为一个$m \times n$矩阵,简记为$A$或$(a_{ij})_{m\times n}(i=1,2,\cdots,m;j=1,2,\cdots,n).$当$m=n$时,称$A$为$n$阶方阵.两个矩阵$A=(a_{ij})_{m \times n},B=(b_{ij})_{s \times k}$,若$m=s,n=k$,则称$A$与$B$为同型矩阵.
    \end{defn}
    \subsection{矩阵的本质}
    假设英语系有98 个女生,2个男生;机械系有95个男生,5个女生.那么可以使用矩阵来表达上述信息.
    $$
        \begin{bmatrix}
            2  & 95 \\
            98 & 5
        \end{bmatrix}
    $$
    上述矩阵第一列表达了英语系的男、女生人数,第二列表达了机械系的男、女生人数,而第一行表达了不同系的男生人数,第二行表达了不同系的女生人数.\textbf{矩阵是用来表达系统信息},具体来说,矩阵通过其元素和排列方式,能够系统地表示出多个变量之间的关系或数据间的结构. 此外还有以下两个观点:
    \begin{enumerate}
        \item 矩阵也是由若干行(列)向量拼成的
        \item 矩阵不能运算,但是其若干行(列)向量之间可能存在着某种关系,这种关系反应了矩阵的本质,即矩阵的秩.\textbf{矩阵秩的本质是组成该矩阵的线性无关的向量的个数(即基向量的个数)}.
    \end{enumerate}
    以上述矩阵为例,可视为矩阵是由$[2,95],[98,5]$两个行向量组成,或者是由$[2,98]^{\mathrm{T}},[95,5]^{\mathrm{T}}$两个列向量组成.此外,以该矩阵为例$$\begin{bmatrix}
            1 & 0 \\
            0 & 1 \\
            2 & 5 \\
            4 & 7
        \end{bmatrix}$$其中$[2,5]^{\mathrm{T}},[4,7]^{\mathrm{T}}$可以由$[1,0]^{\mathrm{T}},[0,1]^{\mathrm{T}}$组成.那么就称$[1,0]^{\mathrm{T}},[0,1]^{\mathrm{T}}$为基向量,向量空间中任意一个元素,都可以唯一地表示成基向量的线性组合.
    \subsection{重要矩阵}
    \begin{enumerate}
        \item 零矩阵:每个元素均为零的矩阵,记为$\boldsymbol{O}$.
        \item 单位矩阵:主对角线元素均为1,其余元素全为零的$n$阶方阵,称为$n$阶单位矩阵,记成$E$(或$I$)
              $$
                  \begin{bmatrix}
                      1      & 0      & \cdots & 0      \\
                      0      & 1      & \cdots & 0      \\
                      \vdots & \vdots & \ddots & \vdots \\
                      0      & 0      & \cdots & 1
                  \end{bmatrix}
              $$
        \item 数量矩阵:数$k$和单位矩阵的乘积称为数量矩阵
              $$
                  \begin{bmatrix}
                      k      & 0      & \cdots & 0      \\
                      0      & k      & \cdots & 0      \\
                      \vdots & \vdots & \ddots & \vdots \\
                      0      & 0      & \cdots & k
                  \end{bmatrix}
              $$
        \item 对角矩阵:非主对角线元素均为零的矩阵称为对角矩阵
              $$
                  \begin{bmatrix}
                      \lambda_1 & 0         & \cdots & 0         \\
                      0         & \lambda_2 & \cdots & 0         \\
                      \vdots    & \vdots    & \ddots & \vdots    \\
                      0         & 0         & \cdots & \lambda_n
                  \end{bmatrix}
              $$
        \item 上(下)三角矩阵:当$i>(<)j$时,$a_{ij}=0$的矩阵称为上(下)三角矩阵
        \item 对称矩阵:满足条件$A^{\mathrm{T}}=A$的矩阵$A$称为对称矩阵,$A^{\mathrm{T}}=A\Leftrightarrow a_{ij}=a_{ji}$
              $$
                  A=\begin{bmatrix}
                      a & b & c \\
                      b & d & e \\
                      c & e & f
                  \end{bmatrix},A^\mathrm{T}=\begin{bmatrix}
                      a & b & c \\
                      b & d & e \\
                      c & e & f
                  \end{bmatrix}
              $$
        \item 反对称矩阵:满足条件$A^\mathrm{T}=-A$的矩阵$A$称为反对称矩阵$A^{\mathrm{T}}=-A\Leftrightarrow\begin{cases}a_{ij}=-a_{ji} ,i\neq j ,\\a_{ii}=0.&\end{cases}$
              $$
                  A=\begin{bmatrix}
                      0 & -b & -c \\
                      b & 0  & -a \\
                      c & a  & 0
                  \end{bmatrix},A^\mathrm{T}=\begin{bmatrix}
                      0  & b  & c \\
                      -b & 0  & a \\
                      -c & -a & 0
                  \end{bmatrix}
              $$
        \item 行矩阵:只有一行元素的矩阵,也称行向量
        \item 列矩阵:只有一列元素的矩阵,也称列向量
        \item gram矩阵:以矩阵$A_{2\times 3}=\begin{bmatrix}
                      \alpha_1 , \alpha_2 , \alpha_3
                  \end{bmatrix}$为例,其中$\alpha_1=(a_{11},a_{21})^{\mathrm{T}},\alpha_2=(a_{12},a_{22})^{\mathrm{T}},\alpha_3=(a_{13},a_{23})^{\mathrm{T}},A_{3 \times 2} ^{\mathrm{T}}=\begin{bmatrix}\alpha_1^\mathrm{T},\alpha_2^\mathrm{T},\alpha_3^\mathrm{T}
                  \end{bmatrix}$,那么$A^{\mathrm{T}}A=\left.\left(
                  \begin{matrix}
                      a_{11} & a_{21} \\
                      a_{12} & a_{22} \\
                      a_{13} & a_{23}
                  \end{matrix}\right.\right)_{3\times2}
                  \left(\begin{matrix}
                      a_{11} & a_{12} & a_{13} \\a_{21} & a_{22} & a_{23}
                  \end{matrix}\right)_{2\times3}$.可以观察到\textbf{$A^{\mathrm{T}}A$是一个方阵}.此外,上式可以接着化简:$\left(\begin{matrix}
                          |\alpha_1||\alpha_1|\cos\theta & |\alpha_1||\alpha_2|\cos\theta & |\alpha_1||\alpha_3|\cos\theta \\
                          |\alpha_2||\alpha_1|\cos\theta & |\alpha_2||\alpha_2|\cos\theta & |\alpha_2||\alpha_3|\cos\theta \\
                          |\alpha_3||\alpha_1|\cos\theta & |\alpha_3||\alpha_2|\cos\theta & |\alpha_3||\alpha_3|\cos\theta
                      \end{matrix}\right)$,其中$\cos \theta$为余弦相似性.余弦相似性是用向量空间中两个向量夹角的余弦值作为衡量两个向量间差异的大小的度量.余弦值越接近1,就表明夹角越接近0度,也就是两个向量越相似.
    \end{enumerate}
    \subsection{分块矩阵}
    用几条纵线和横线把一个矩阵分成若干小块,每一小块称为原矩阵的子块,把子块看作原矩阵的一个元素,就得到了分块矩阵.如 $\boldsymbol{A}$ 按行分块:
    $$
        \boldsymbol{A}=\left[\begin{array}{cccc}
                a_{11}             & a_{12}  & \cdots & a_{1 n} \\
                \hdashline a_{21}  & a_{22}  & \cdots & a_{2 n} \\
                \hdashline \vdots  & \vdots  &        & \vdots  \\
                \hdashline a_{m 1} & a_{m 2} & \cdots & a_{m n}
            \end{array}\right]=\left[\begin{array}{c}
                A_1    \\
                A_2    \\
                \vdots \\
                A_m
            \end{array}\right],
    $$
    其中,$\boldsymbol{A}_i=\left[a_{i 1},a_{i 2},\cdots,a_{i n}\right](i=1,2,\cdots,m)$ 是 $\boldsymbol{A}$ 的一个子块.
    $B$ 按列分块:
    $$
        \boldsymbol{B}=\left[\begin{array}{c:c:c:c}
                b_{11}  & b_{12}  & \cdots & b_{1 n} \\
                b_{21}  & b_{22}  & \cdots & b_{2 n} \\
                \vdots  & \vdots  &        & \vdots  \\
                b_{m 1} & b_{m 2} & \cdots & b_{m n}
            \end{array}\right]=\left[\boldsymbol{B}_1,\boldsymbol{B}_2,\cdots,\boldsymbol{B}_n\right],
    $$
    其中,$B_j=\left[b_{1 j},b_{2 j},\cdots,b_{m j}\right]^{\mathrm{T}}(j=1,2,\cdots,n)$ 是 $B$ 的的一个子块.

    \section{矩阵的基本运算}
    \begin{enumerate}
        \item 相等:$A= ( a_{ij}) _{m\times n}= B= ( b_{ij}) _{s\times k}\Leftrightarrow m= s,n= k$,且$a_{ij}=b_{ij}(i=1,2,\cdots,m;j=1,2,\cdots,n)$,即$A,B$是同型矩阵,且对应元素相等.
        \item 加法:两个矩阵是同型矩阵时,可以相加,即$C=A+B=\left(a_{ij}\right)_{m\times n}+\left(b_{ij}\right)_{m\times n}=\left(c_{ij}\right)_{m\times n}$,其中,$c_{ij}=a_{ij}+b_{ij}\left(i=1,2,\cdots,m;j=1,2,\cdots,n\right)$,即对应元素相加.
        \item 数乘矩阵:设$k$是一个数,$A$是一个$m \times n$矩阵.数$k$和$A$的乘积称为数乘矩阵.即:
              $$
                  kA=Ak=k\begin{bmatrix}a_{11}&a_{12}&\cdots&a_{1n}\\a_{21}&a_{22}&\cdots&a_{2n}\\\vdots&\vdots&&\vdots\\a_{m1}&a_{m2}&\cdots&a_{mn}\end{bmatrix}=\begin{bmatrix}ka_{11}&ka_{12}&\cdots&ka_{1n}\\ka_{21}&ka_{22}&\cdots&ka_{2n}\\\vdots&\vdots&&\vdots\\ka_{m1}&ka_{m2}&\cdots&ka_{mn}\end{bmatrix}=(ka_{ij})_{m \times n}
              $$即$A$的每个元素都乘以$k$.
              \begin{enumerate}
                  \item 交换律:$A+B=B+A$
                  \item 结合律:$(A+B)+C=A+(B+C)$
                  \item 分配律:$k(A+B)=kA+kB,(k+l)A=kA+lA$
                  \item 数和矩阵相乘的结合律:$k(lA)=(kl)A=l(kA)$
              \end{enumerate}
        \item 矩阵乘法:设$A$是$m \times s$矩阵,$B$是$s \times n$矩阵(矩阵$A$的列数必须与矩阵 $B$ 的行数相等),则$A$ , $B$可以相乘,乘积$AB$ 是$m\times n$矩阵,记$C=AB=(c_{ij})_\mathrm{m \times m}.C$ 的第$i$行第$j$列元素$c_{ij}$ 是$A$的第$i$行的$s$个元素与$B$的第$j$列的$s$个对应元素两两乘积之和,即$$c_{ij}=\sum_{k=1}a_{ik}b_{kj}=a_{i1}b_{1j}+a_{i2}b_{2j}+\cdots+a_{is}b_{sj}(i=1,2,\cdots,m; j=1, 2,\cdots, n)$$
              \begin{enumerate}
                  \item 结合律:$(A_{m\times s}B_{s\times r})C_{r\times n}=A_{m\times s}(B_{s\times r}C_{r\times n})$
                  \item 分配律:$A_{m\times s}(B_{s\times n}+C_{s\times n})=A_{m\times s}B_{s\times n}+A_{m\times s}C_{s\times n}$
                  \item 数和矩阵相乘的结合律:$(A_{m\times s}+B_{m\times s})C_{s\times n}=A_{m\times s}C_{s\times n}+B_{m\times s}C_{s\times n}$
              \end{enumerate}
              \begin{criterion}{关于矩阵乘法下$AB\neq BA$}{}
                  \begin{itemize}
                      \item 矩阵的乘法一般情况下不满足交换律,即$AB\neq BA.$例如
                            $$A=\begin{bmatrix}1&1\\-1&-1\end{bmatrix},\:B=\begin{bmatrix}1&-1\\-1&1\end{bmatrix},$$则$$AB=\begin{bmatrix}1&1\\-1&-1\end{bmatrix}\begin{bmatrix}1&-1\\-1&1\end{bmatrix}=\begin{bmatrix}0&0\\0&0\end{bmatrix},$$$$BA=\begin{bmatrix}1&-1\\-1&1\end{bmatrix}\begin{bmatrix}1&1\\-1&-1\end{bmatrix}=\begin{bmatrix}2&2\\-2&-2\end{bmatrix},\:AB\neq BA\:.$$
                      \item 由上面的例子知,存在$A\neq O,B\neq O$,而$AB=O$的情况,故$AB=O\nRightarrow A=O$或$B=O$.
                      \item $AB=AC\Rightarrow A(B-C)=O$,此时即使有$A\neq O$,一般也得不出$B=C$.
                  \end{itemize}
              \end{criterion}
        \item 转置矩阵:将$m \times n$矩阵$A=(a_{ij})_{m\times n}$的行与列互换得到的$n\times m$矩阵,称为矩阵$A$的转置矩阵,记为$A^\mathrm{T}$,即$$A^\mathrm{T}=\begin{bmatrix}a_{11}&a_{21}&\cdots&a_{m1}\\a_{12}&a_{22}&\cdots&a_{m2}\\\vdots&\vdots&&\vdots\\a_{1n}&a_{2n}&\cdots&a_{mn}\end{bmatrix}.$$转置矩阵满足下列运算规律:
              \begin{enumerate}
                  \item $(A^{\mathrm{T}})^{\mathrm{T}}=A$
                  \item $(kA)^{\mathrm{T}}=kA^{\mathrm{T}}$
                  \item $(A+B)^{\mathrm{T}}=A^{\mathrm{T}}+B^{\mathrm{T}}$
                  \item \textcolor{red}{$(AB)^{\mathrm{T}}=B^{\mathrm{T}}A^{\mathrm{T}}$}
              \end{enumerate}
        \item 方阵的幂:$A$是一个$n$阶方阵,$A^m=\overbrace{AA\cdots A}^{m\text{个}}$称为$A$的$m$次幂.
              \begin{criterion}{关于方阵的幂的注意事项}{}
                  \begin{enumerate}
                      \item 因大部分情况下$AB \neq BA$,所以没有下面的式子:\newline$(A+B)^{2}=(A+B)(A+B)=A^{2}+AB+BA+B^{2}\neq A^{2}+2AB+B^2$ ,\newline$(A-B)^{2}=A^{2}-AB-BA+B^{2} \neq A^{2}-2AB+B^{2}$,\newline$(A+B)(A-B)=A^{2}+BA-AB-B^{2}\neq A^{2}-B^{2}$,\newline$(AB)^m=\overbrace{(AB)(AB)\cdots (AB)}^{m\text{个}}\neq A^m B^m$
                      \item 在微积分中可知,函数的表达式可以由该函数的高阶导数表达式构造出来,那么就有:$f(x)=a_0+a_1x+\cdots+a_mx^m$,将其中的未知数$x$替换为$n$阶方阵$A$可得:\textbf{$f(A)=a_0E+a_1A+\cdots+a_mA^m$}
                  \end{enumerate}
              \end{criterion}
        \item 方阵的行列式:当用$n$阶方阵$A$计算行列式时,记成$|A|$.
              \begin{criterion}{$n$阶方阵计算行列式的注意事项}{}
                  \begin{enumerate}
                      \item $\left|kA\right|=k^{n}\left|A\right|\neq k\left|A\right|\left(n\geqslant2, k\neq0, 1\right)$
                      \item 一般地,$\left|A+B\right|\neq\left|A\right|+\left|B\right|$
                      \item $A\neq O\Rightarrow\left|A\right|\neq 0$ \footnote{如:$A=\begin{bmatrix}1 & 0 \\ 0 & 0\end{bmatrix}$}
                      \item $A\neq B\nRightarrow\left|A\right|\neq\left|B\right|$
                      \item $\left|A^{\mathrm{T}}\right|=\left|A\right|$ \footnote{行列式行列互换值不变}
                      \item 设$A,B$是同阶方阵,则$|AB|=|A\|B|$
                  \end{enumerate}
              \end{criterion}
        \item 分块矩阵的基本运算(以$2 \times 2$型分块矩阵为例):
              \begin{enumerate}
                  \item 加法:同型,且分法一致,则$\begin{bmatrix}A_1&A_2\\A_3&A_4\end{bmatrix}+\begin{bmatrix}B_1&B_2\\B_3&B_4\end{bmatrix}=\begin{bmatrix}A_1+B_1&A_2+B_2\\A_3+B_3&A_4+B_4\end{bmatrix}$
                  \item 数乘:$k{\begin{bmatrix}A&B\\C&D\end{bmatrix}}={\begin{bmatrix}kA&kB\\kC&kD\end{bmatrix}}$
                  \item 乘法:$\begin{bmatrix}A&B\\C&D\end{bmatrix}\begin{bmatrix}X&Y\\Z&W\end{bmatrix}=\begin{bmatrix}AX+BZ&AY+BW\\CX+DZ&CY+DW\end{bmatrix}$,要可乘、可加
                  \item 若$A,B$分别为$m,n$阶方阵,则分块对角矩阵的幂为
                        $$\begin{bmatrix}A&O\\O&B\end{bmatrix}^n=\begin{bmatrix}A^n&O\\O&B^n\end{bmatrix}$$
              \end{enumerate}
    \end{enumerate}
    \section{逆矩阵}
    \begin{defn}{逆矩阵的定义}{}
        $A,B$是$n$阶\textbf{方阵},$E$是$n$阶单位矩阵,若$AB=BA=E$,则称$A$是可逆矩阵,并称$B$是$A$的逆矩阵,且逆矩阵是唯一的,记作$A^{-1}$.
        \tcblower
        $A$ 可逆的充分必要条件是$|A|\neq0$ \footnote{行列式的值不为0}
    \end{defn}
    \begin{criterion}{逆矩阵的性质与重要公式}{}
        设A,B是同阶可逆方阵,则
        \begin{enumerate}
            \item $(A^{-1})^{-1}=A$
            \item 若$k\neq 0$, 则$(kA)^{- 1}=\dfrac1k A^{- 1}$
                  \begin{proof}
                      $(\dfrac{1}{k} \times k) (A \times A^{-1})=E$
                  \end{proof}
            \item $AB$ 也可逆,且$(AB)^-1=B^{-1}A^{-1}$ \footnote{“穿脱”原则}
            \item $A^{\mathrm{T}}$也可逆,且$(A^{\mathrm{T} })^{-1}=(A^{- 1})^{\mathrm{T}}$
                  \begin{proof}
                      $A^{\mathrm{T}} \cdot (A^{\mathrm{T}})^{-1}=E$\newline
                      $A^{\mathrm{T}} \cdot (A^{-1})^{\mathrm{T}}=E$\newline
                      $(A^{-1}\cdot A)^{\mathrm{T}}=E$
                  \end{proof}
            \item $\left|A^{-1}\right|=\left|A\right|^{-1}$
                  \begin{proof}
                      $ \because |AB|=|A|\cdot|B|$
                      $ \therefore  |A^{-1} \cdot A|=|E|=|A^{-1}||A|=1$
                  \end{proof}
        \end{enumerate}
    \end{criterion}
    \begin{criterion}{用定义法求可逆矩阵的逆矩阵}{}
        \begin{itemize}
            \item 依定义进行求解,即求一个矩阵$B$,使$AB=E$,则$A$可逆,且$A^{-1}=B$.
            \item 将$A$分解成若干个可逆矩阵的乘积.因两个可逆矩阵的积仍是可逆矩阵,即若$A=BC$,其中,$B,C$均可逆,则$A$可逆,且$A^{-1}=(BC)^{-1}=C^{-1}B^{-1}$.
        \end{itemize}
    \end{criterion}
    \section{伴随矩阵}
    \begin{defn}{伴随矩阵的定义}{}
        将行列式$|A|$的$n^2$个元素的代数余子式按如下形式排成的矩阵称为$A$的伴随矩阵,记作$A^*$,即
        $$A^{*}=
            \begin{bmatrix}
                A_{11} & A_{21} & \cdots & A_{n1} \\
                A_{12} & A_{22} & \cdots & A_{n2} \\
                \vdots & \vdots &        & \vdots \\
                A_{1n} & A_{2n} & \cdots & A_{nn}
            \end{bmatrix}
        $$,且有:$AA^{*}=A^{*}A=|A|E$.
    \end{defn}
    \textbf{根据伴随矩阵的定义,需要注意的是:$A_{ij}$在$A^*$中的位置,要确保$a_{ij} \cdot A_{ij}$,这样最后$AA^*$的值才是$|A|E$.}
    \begin{criterion}{伴随矩阵的性质与重要公式}{}
        \begin{enumerate}
            \item 对任意$n$阶方阵$A$,都有伴随矩阵$A^*$,且有公式$AA^{*}=A^{*}A=\left|A\right|E,\left|A^{*}\right|=\left|A\right|^{n-1}$.\newline
                  当$|A|\neq 0$时,有
                  \begin{itemize}
                      \item $A^{*}=\left|A\right|A^{-1},\:A^{-1}=\dfrac{1}{\left|A\right|}A^{*},\:A=\left|A\right|(A^{*})^{-1}$ \footnote{该条性质说明了当$A$可逆时,$A^*$与$A^{-1}$只差了一个常数倍$|A|$,那也就是说二者性质相同}
                            \begin{proof}
                                $\because A A^*=|A|E \newline \therefore A^{-1}AA^*=A^{-1}|A|E \newline \therefore A^*=|A|A^{-1}$
                            \end{proof}
                      \item $(kA)(kA)^*=|kA|E$\newline
                            以下三个公式都是$AA^*=|A|E$的推广
                      \item $A^{\mathrm{T}}(A^{\mathrm{T}})^{\mathrm{*}}=\left|A^{\mathrm{T}}\right|E$
                      \item $A^{-1}(A^{-1})^*=\left|A^{-1}\right|E$
                      \item $A^*(A^*)^*=\left|A^*\right|E$
                  \end{itemize}
            \item $(A^{\mathrm{T}})^{*}=(A^{*})^{\mathrm{T}},(A^{-1})^{*}=(A^{*})^{-1}, (AB)^{*}=B^{*}A^{*},(A^{*})^{*}=|A|^{n-2}A.$
        \end{enumerate}
    \end{criterion}
    \section{初等变换与初等矩阵}
    \begin{defn}{初等变换}{}
        \begin{enumerate}
            \item 一个非零常数乘矩阵的某一行(列)
            \item 互换矩阵中某两行(列)的位置
            \item 将矩阵的某一行(列)的$k$倍加到另一行(列)
        \end{enumerate}
        以上三种变换称为矩阵的初等行(列)变换,且分别称为倍乘、互换、倍加初等行(列)变换。
    \end{defn}
    \begin{defn}{初等矩阵}{}
        由\textbf{单位矩阵}经过一次初等变换得到的矩阵称为\textbf{初等矩阵}.
        \begin{itemize}
            \item  $\boldsymbol{E}_2(k)=\left[\begin{array}{lll}1 & 0 & 0 \\ 0 & k & 0 \\ 0 & 0 & 1\end{array}\right],E$的第2行(或第2列)乘$k$倍,称为倍乘初等矩阵。定义:$E_i(k)(k \neq 0)$ 表示单位矩阵 $\boldsymbol{E}$ 的第 $i$ 行(或第 $i$ 列)乘以非零常数 $k$ 所得的初等矩阵。
            \item $\boldsymbol{E}_{12}=\left[\begin{array}{lll}0 & 1 & 0 \\ 1 & 0 & 0 \\ 0 & 0 & 1\end{array}\right],\boldsymbol{E}$的第1,2行(或第1,2列)互换,称为互换初等矩阵。定义:$E_{i j}$ 表示单位矩阵 $\boldsymbol{E}$ 交换第 $i$ 行与第 $j$ 行(或交换第 $i$ 列与第 $j$ 列)所得的初等矩阵。
            \item $\boldsymbol{E}_{31}(k)=\left[\begin{array}{lll}1 & 0 & 0 \\ 0 & 1 & 0 \\ k & 0 & 1\end{array}\right], \boldsymbol{E}$ 的第 1 行的 $k$ 倍加到第 3 行(或第3列的$k$倍加到第1列),称为倍加初等矩阵.定义:$E_{i j}(k)$ 表示单位矩阵 $\boldsymbol{E}$ 的第 $j$ 行的 $k$ 倍加到第 $i$ 行(或第 $i$ 列的 $k$ 倍加到第 $j$ 列)所得的初等矩阵。
        \end{itemize}
    \end{defn}
    \begin{defn}{初等矩阵的性质与重要公式}{}
        \begin{itemize}
            \item 初等矩阵的转置仍是初等矩阵
            \item 因为$\left|E_i(k)\right|=k\neq0,\left|E_{ij}\right|=-1\neq0,\left|E_{ij}(k)\right|=1\neq0$,故初等矩阵都是可逆矩阵。且$\left[E_i(k)\right]^{-1}=E_i\left(\dfrac{1}{k}\right), E_{ij}^{-1}=E_{ij} ,\left[E_{ij}(k)\right]^{-1}=E_{ij}(-k)$,其逆矩阵仍是同一类型的初等矩阵。
            \item 若$A$是可逆矩阵,则$A$可以表示成有限个初等矩阵的乘积,即$A=P_1P_2\cdots P_s$, 其中 $P_1P_2\cdots P_s$ 是初等矩阵。
            \item 若$A$可逆,则$A$一定可以通过若干(有限)次初等行变换化为同阶单位矩阵$E$,即$Q_s\cdots Q_2Q_1A=E\mathrm{~.}$
            \item \textbf{对$n$阶矩阵$A$进行初等行变换,相当于在矩阵$A$的左边乘相应的初等矩阵.同样,对$A$进行初等列变换,相当于在矩阵$A$的右边乘相应的初等矩阵}。\footnote{行变换:$\begin{bmatrix}0 & 1\\ 1 & 0 \end{bmatrix}\begin{bmatrix}1 & 2\\ 3 & 4 \end{bmatrix}=\begin{bmatrix}3 & 4\\ 1 & 2 \end{bmatrix}$。列变化:$\begin{bmatrix}1 & 2\\ 3 & 4 \end{bmatrix}\begin{bmatrix}0 & 1\\ 1 & 0 \end{bmatrix}=\begin{bmatrix}2 & 1\\ 4 & 3 \end{bmatrix}$}
        \end{itemize}
    \end{defn}
    \begin{defn}{行阶梯形矩阵}{}
        具有如下特征的矩阵称为\textbf{行阶梯形矩阵}:
        \begin{enumerate}
            \item 若有零行(即元素全为零的行),则零行全都位于非零行的下方
            \item 各非零行左起第一个非零元素的列指标由上至下是严格增大的
        \end{enumerate}\footnote{如:$\begin{bmatrix} 1 & 1 & 0 & -3 & -1 \\ 0 & -2 & -2 & 2 & 1\\ 0 & 0 & 0 & 3 & -1 \\  0 & 0 & 0 & 0 & 0  \end{bmatrix}$,这就是行阶梯形矩阵。}
    \end{defn}
    \begin{defn}{行最简阶梯形矩阵}{}
        一个\textbf{行阶梯形矩阵}称为\textbf{行最简阶梯形矩阵},如果其非零行的第一个非零元素为1,并且这些非零元素所在列的其他元素均为0。
    \end{defn}
    \begin{criterion}{用初等变换求逆矩阵的方法}{}
        $$[\boldsymbol{A} \vdots \boldsymbol{E}] \xrightarrow{\text { 初等行变换 }}\left[\boldsymbol{E} \vdots \boldsymbol{A}^{-1}\right]$$
        $$\left[\begin{array}{l}\boldsymbol{A} \\ \boldsymbol{E}\end{array}\right] \xrightarrow{\text { 初等列变换 }}\left[\begin{array}{c}\boldsymbol{E} \\ \boldsymbol{A}^{-1}\end{array}\right]$$
    \end{criterion}
    \begin{criterion}{简单分块矩阵的逆}{}
        若$A,B$均是可逆方阵,则
        $$
            \begin{bmatrix}A&O\\O&B\end{bmatrix}^{-1}=\begin{bmatrix}A^{-1}&O\\O&B^{-1}\end{bmatrix},\begin{bmatrix}O&A\\B&O\end{bmatrix}^{-1}=\begin{bmatrix}O&B^{-1}\\A^{-1}&O\end{bmatrix}.
        $$
    \end{criterion}
    \section{矩阵方程}
    \begin{defn}{矩阵方程的定义}{}
        含有未知矩阵的方程称为矩阵方程。\newline
        解矩阵方程,应先根据题设条件和矩阵的运算规则,将方程进行恒等变形,使方程化成$AX=B$,$XA=B$或$AXB=C$的形式.\newline
        若$A$可逆,或$A,B$均可逆,则分别可得解为$X=A^{-1} B,X=BA^{-1},X=A^{-1}CB^{-1}.$
    \end{defn}
    \section{等价矩阵和矩阵的等价标准形}
    \begin{defn}{等价矩阵的定义}{}
        设 $\boldsymbol{A}, \boldsymbol{B}$ 均是 $\boldsymbol{m} \times \boldsymbol{n}$ 矩阵, 若存在可逆矩阵 $\boldsymbol{P}_{m \times m}, \boldsymbol{Q}_{n \times n}$, 使得 $\boldsymbol{P A Q}=\boldsymbol{B}$, 则称 $\boldsymbol{A}, \boldsymbol{B}$ 是等价矩阵, 记作 $A \cong B$。
    \end{defn}
    \begin{defn}{矩阵的等价标准形}{}
        $A$是一个$m\times n$矩阵,则$A$等价于形如$\begin{bmatrix}\boldsymbol{E},&\boldsymbol{0}\\\boldsymbol{0}&\boldsymbol{0}\end{bmatrix}$的矩阵($E_r$中的$r$恰是$r(A))$,后者称为$A$的等价标准形。等价标准形是唯一的,即若$r(A)=r$,则存在可逆矩阵$P,Q$使得$$PAQ=\begin{bmatrix}E_r&O\\O&O\end{bmatrix}$$
    \end{defn}
    \section{矩阵的秩}
    \begin{defn}{矩阵秩的定义}{}
        设$A$是$m\times n$矩阵,若存在$k$阶子式\footnote{任取$k$行,$k$列构成$k$阶子式}不为零,而任意$k+1$阶子式(如果有的话)全为零\footnote{$k$维体积不为0,$k$维下线性无关。$k+1$维体积为0,$k+1$维下线性相关},则$r(A)=k$,且若$A$为$n\times n$矩阵,则
        $$
            r(A_{n\times n})=n\Leftrightarrow|A|\neq0\Leftrightarrow A\text{ 可逆 }.
        $$
    \end{defn}
    \begin{conclusion}{矩阵秩的相关结论}{}
        设$A$是$m\times n$矩阵,$B$是满足有关矩阵运算要求的矩阵,则
        \begin{itemize}
            \item $0 \leqslant r(A) \leqslant \min \{m, n\}$ (由定义)
            \item $r(k A)=r(A)(k \neq 0)$ ( 由定义 )
            \item $r(\boldsymbol{A} \boldsymbol{B}) \leqslant \min \{r(\boldsymbol{A}), r(\boldsymbol{B})\}$
            \item $r(\boldsymbol{A}+\boldsymbol{B}) \leqslant r(\boldsymbol{A})+r(\boldsymbol{B})$
            \item $r\left(\boldsymbol{A}^*\right)= \begin{cases}n,r(A)=n\\1,  r(\boldsymbol{A})=n-1, \\ 0,  r(\boldsymbol{A})<n-1\end{cases}$,其中$A$为$n(n≥2)$阶方阵
            \item 设 $\boldsymbol{A}$ 是 $m \times n$ 矩阵, $\boldsymbol{P}, \boldsymbol{Q}$ 分别是 $m$ 阶、$n$ 阶可逆矩阵,则
                  $$r(\boldsymbol{A})=r(PA )=r(\boldsymbol{A Q})=r(PAQ)$$
            \item 若$\boldsymbol{A}_{m \times n} \boldsymbol{B}_{n \times s}=\boldsymbol{O}$, 则 $r(\boldsymbol{A})+r(\boldsymbol{B}) \leqslant n$
            \item $r(\boldsymbol{A})=r\left(\boldsymbol{A}^{\mathrm{T}}\right)=r\left(\boldsymbol{A}^{\mathrm{T}} \boldsymbol{A}\right)=r\left(\boldsymbol{A} \boldsymbol{A}^{\mathrm{T}}\right)$
        \end{itemize}
    \end{conclusion}
    %  ############################ 正文部分
    \ifx\allfiles\undefined
\end{sloppypar}
\end{document}
\fi
